\section*{Introduction and Context}

Perhaps it could be said that functional programmers are well aware of the influence of category theory in their trade, most notably it’s influence on modeling computational `effects', and more broadly, cleverly abstracting context and state between function calls. 

Yet -as of this writing- there is talk of utilizing the ongoing development contained in category theory more broadly, and throughout the various STEM fields. Furthermore so that e.g. ad-hock models developed in one field, may be applicable in more diverse settings. As written in a post on the ACT website,
\begin{quotation}
``we believe that category theory has the potential to bridge specific different fields, and moreover that developments in such fields (e.g. automata) can be transferred successfully into other fields (e.g. systems biology) through category theory.''
\textsuperscript{(ACT 2018)}
\end{quotation}

In this context, my paper of concern is by Elaine Landry, called ``Category Theory: The Language of Mathematics''. Written in 1999, the author sets out her premise for the role of Category Theory in mathematics, and mathematical discourse. \textsuperscript{(Landry)}

\section*{Motivation}

My topic is about a debate in the mathematical community and therein what should be considered the foundation of all mathematics and other standardization concerns. So therefore I presume that I will need to convince my audience why you should care.

You should care because the future of computing applications -those that may change your life in perhaps unfathomable ways- will depend on more computationally efficient models to be viable in any significant manner. What kinda models? Well let me enlighten you of a microcosm example.

There is this very interesting video essay by the Heme Review Podcast (i.e.  Dr. Bernard) called ``A Future Of Medicine''. In this essay, Dr. Bernard explains how our current medial system works and some more ideal hypothetical system, respectively. What is important, and the thesis of his essay, are the flaws of our current medical system (which he explains in exquisite detail). Overall, what is interesting -I think- is how our modern evidence based system isn’t as ideal as I once thought, and this is due to it’s dependence on population data, so such may not be directly applicable to a given person. For example, if a trial was done on a bunch of men, the speaker said clinical experience suggests that there will be differences for women taking the same medication. Whereas, his proposed hypothetical system disregards the need for such models VIA personalized simulations of a given individual, and I suppose all their biological processes. I.e. if this technology was available, there would be no need to study the efficacy VIA population models, because with this hypothetical technology, such simulations would be far more accurate, would be far more personalized to the biochemistry of the given individual, and not from generalizations of population data. Furthermore, if possible, he said, such could likewise -hypothetically speaking- be extended to drug development VIA optimization algorithms. \textsuperscript{(Bernard)}

This is why you should care, and this is but a microcosm example of what may be possible (something that simply isn't possible nowadays). Furthermore, you may be thinking that the solution is more computing power. But I think this may be unlikely compared to more fundamental developments in the underlying models of such complex systems. As a slide by the Air Force Research Laboratory suggests, ``the curse of dimensionality won’t be mitigated fully by more computing power [forward arrow] more math modeling is needed'' \textsuperscript{(Fahroo)}.


\section*{Category Theory: The Language of Mathematics}

In the beginning, Elaine Landry notes three predominate stances in the literate, notably being that category theory is foundational to mathematics, that it is not -in favor of set theory- or that category theory provides an ``organizational role''.  Thereafter she proposes her own claim that, ``category theory provides the language for mathematical discourse.''

Furthermore, she argues that category theory is assigned, ``the role of organizing our talk of the structure of mathematical concepts and theories''. Here I think she is referring to category theory as essentially defining a taxonomy of some sort. Perhaps akin to the terminology Haskell adopted from Category Theory, and the manner in which such is conveyed. 

As a programmer, her remark of ``mathematical discourse'' is interesting. It is as though she is defining the language from which mathematical structures will be defined, expressed, and communicated. Furthermore, perhaps akin to standardization concerns that we see in engineering disciplines.

She argues that a foundation in mathematics, ``ought to be seen as accounting for both the classificatory and eliminatory role of theories''. In contrast to, and that ``a foundation need not be seen as providing either the form or the content of mathematics'', as I would have expected. But again, she frequently remarks of this `protean' nature to mathematics. As in, given it’s very nature, standardizing on any given implementation isn’t possible nor ideal. Therefore we have to consider what we mean by foundational in a different manner. 

Furthermore as a programmer, her remark is rather interesting, because if anything, programming is exceptionally protean in nature. Her remark that ``a foundation need not be seen as providing either the form or the content of mathematics'', would be akin to computer scientists disregarding modern developments in favor of some preexisting medium of the current time. Or put another way, akin to standardizing on assembly language (which would drastically alter the software industry because of the economics of the industry is different, nowadays software developers are generally expensive, while hardware is generally cheap). Though regardless of any specific implementation, some ideas in programming have been around for a long time. For instance lambda calculus was developed sometime in early 1900’s, and the foundational terminology therein hasn’t changed (at least to my knowledge).

Furthermore concerning this `protean' nature intrinsic mathematics, she therefore asserts that, 
\begin{quotation}
``category theory reflects the protean character of mathematics by providing the means for organizing what we say in such a manner that allows us to talk about both mathematical structures and the structure of such structures.''
\end{quotation}

As in, category theory provides the language for communicating mathematical structure. Yet quite unlike anything from set theory. Because while set theory is concerned with elements, category theory abstracts such elements entirely into e.g. the image of a given mapping, and therefore considers the ultimate structure of things in a more distilled manner. This gives way to the finale -rather poetic- conclusion of her paper, as she wrote,
\begin{quotation}
``Our lesson then is this: just as mathematics is protean with respect to the empirical or scientific world, so category theory is protean with respect to mathematical discourse. Just as mathematics can be seen as providing the language for the world —it allows us to talk about physical objects in structural terms without having to be about those objects— so category theory can be seen as providing the language for mathematics —it allows us to talk about objects in structural terms without having to be about those objects.''
\end{quotation}

\section*{Forward}

Nowadays category theory is getting considerable interest in various fields, so much so that it makes me wonder what the environment was like when the paper was written. At such a time, would the popular interest in such a field be unimaginable to the perhaps small coalition of researches from 1999?

I don’t know what the author would make of things nowadays. But regardless, her arguments have certainly manifested in some notable fashions. As Tai-danae Bradley wrote,
\begin{quotation}
``More recently, however, category theory has found applications in a wide range of disciplines outside of pure mathematics—even beyond the closely related fields of computer science and quantum physics. These disciplines include chemistry, neuroscience, systems biology, natural language processing, causality, network theory, dynamical systems, and database theory to name a few.'' \textsuperscript{(BRADLEY)}
\end{quotation}

Perhaps this is an indication of a sound thesis and rationale. 


\section*{Conclusion}

The author was against regarding category theory as a foundation of mathematics due to the ``protean character of mathematics''. To me, her thesis was akin to standardization concerns from engineering disciplines, for she argued that ``category theory provides the language for mathematical discourse''. Though a seemingly simple proposition, a seemingly successful one, given the ongoing manifestation of such.