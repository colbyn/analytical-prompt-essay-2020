\subsection*{Open Source Software}

\subsubsection*{Meta}
The following essay has some history to it. Originally, this was written for an english essay last summer. 

Some bits have been improved and others removed. But you can find the original version over here \url{https://github.com/colbyn/DC-Essay}.


\subsubsection*{Open Source Software}


I propose that there exists a community quite unlike anything you have probably seen from past students. Still in it’s infancy, this community has nevertheless become a foundation of the IT industry. As it matures it will become a cornerstone of efficiency gains in all STEM fields. As our collective knowledge of all technical matters get reduced to mere functions -mapping problems to solutions- this community will become the librarian, troubleshooter, curator, documenter and coordinator of such knowledge.

This community -the printing press of computable knowledge- is the rather humble open source community and industry.

For a brief overview for the layperson: the open source community -as the name implies- is a community built upon open source software, where open source software may be regarded as software built from publicly accessible source code. The open source community may be regarded as the ‘human aspect’ of open source software. As in communication facilities between the producers and consumers of such software. As well as mechanisms for updating the source code that implements such software.

My journey in the open source industry began when I worked at a former startup called upLynk (now absorbed into what's called Version Digital Media Services). At upLynk there was a rule of only using free and open source software, as is common in the IT industry. This wasn’t necessarily because it’s free (the company was immensely profitable as is the industry overall), but due to a multitude of factors. For instance, open source software can be a means of industry standardization. Furthermore, for software that upLynk replied upon, my boss even encouraged me to submit any bug fixes I found in such software back to the original author. This is because it’s generally in the companies interest to do so, because otherwise their version of this software component will be ‘out of sync’ with the official version, so any updates from the ‘official version’ will therefore need to be merged into this nonstandard version, if they wish to benefit from such updates.

To explain, consider the open source ecosystem as system where discrete units build upon other discrete units and therein produce more complex non-discrete units. This system permits for abstraction, so complex non-discrete units may be abstracted into simple discrete units. Thereafter this process of production and abstraction enables further production and abstraction and so forth. Each iteration or generation may be considered to be more sophisticated than prior generations, given that each generation is a product of prior generations… From this analogy, you can imagine these bottom-up and cumulative processes will eventually give rise to very sophisticated products, and perhaps one day, akin to how emergence gives rise to the complexity found in nature.

From personal experience, my \url{https://imager.io} project wouldn’t be possible without the various open source components it’s built upon. Simply because my time is finite, and especially because lower-level encoding details are just \textbf{too complicated for me to understand and implement on my own}. I am nevertheless able to compose such components into a larger and more sophisticated end product, from the preexisting output of resources and information from the global open source community. Overall added value that may be considered to be greater than the sum of its components, and therefore emergent in a manner of speaking. In an old English paper I likened the open source community as ``the printing press of computable knowledge'', and perhaps even more significant than the advent of the printing press itself, because as the industrial revolution introduced a force multiplier of human muscle, so too does abstraction introduce a force multiplier of the human mind. But I digress...  


Because while a book may describe a life’s work in mathematics and abstractions therein, the medium is itself rather passive. A book may describe a life’s work in applied mathematics, yet a mind is required to manifest its application. Whereas, imagine a medium where the most knowledgeable of experts can record their understanding of a given domain as functions that map problems to solutions, in a manner that can be utilized by any layperson, and thereafter this record can be reapplied, reused, and so forth, forever thereafter. This is software, while the open source community is what facilitates the overall adoption, use, and further development of such products. 

Furthermore I propose that this is significant -to society and therein ourselves- because humanity is quite finite, and it’s production capacity is likewise finite, and since humanity must produce that which it wants, humanity is therefore bounded. We are just as bottlenecked as the abstract processes in my model. Nevertheless just as the industrial revolution introduced a force multiplier of human muscle, so too does abstraction introduce a \textbf{force multiplier of the human mind}.

So in conclusion: I propose that there exists a community quite unlike anything you have probably seen from past students and furthermore this discourse community will effect your life in perhaps perhaps unimaginable ways, and this community is the open source community and therein: the production of abstract intellectual goods, when facilitated by the open source community, facilitates emergent phenomena that therein leads to greater sophistication of such output over time. 


\begin{center}
Given the topic, I suppose it goes without saying that this essay has been open sourced, and is available on GitHub:
\url{https://github.com/colbyn/analytical-prompt-essay-2020}.

Likewise, the original version can be found over here: \url{https://github.com/colbyn/DC-Essay}.
\end{center}






