\subsection*{Technical Computing \& Simulations}

There is this very interesting video essay by the Heme Review Podcast (i.e.  Dr. Bernard) called ``A Future Of Medicine''. In this essay, Dr. Bernard explains how our current medial system works and some more ideal hypothetical system, respectively. What is important, and the thesis of his essay, are the flaws of our current medical system (which he explains in exquisite detail). Overall, what is interesting -I think- is how our modern evidence based system isn’t as ideal as I once thought, and this is due to it’s dependence on population data, so such may not be directly applicable to a given person. For example, if a trial was done on a bunch of men, the speaker said clinical experience suggests that there will be differences for women taking the same medication. Whereas, his proposed hypothetical system disregards the need for such models VIA personalized simulations of a given individual, and I suppose all their biological processes. I.e. if this technology was available, there would be no need to study the efficacy VIA population models, because with this hypothetical technology, such simulations would be far more accurate, would be far more personalized to the biochemistry of the given individual, and not from generalizations of population data. Furthermore, if possible, he said, such could likewise -hypothetically speaking- be extended to drug development VIA optimization algorithms. \textsuperscript{(Bernard)}


The above review is a fragment from an old english essay last summer. The video itself is fascinating! (I even showed it to my Mom and she liked it.) Overall, I wish more of such content would get produced and discussed.

